\section{Solicitaciones}

\subsection{Cargas muertas}

Además de considerarse como cargas muertas el peso de las estructuras, techumbres, pisos, muros y paneles, etc., las presiones laterales y verticales de líquidos, gases y materiales fluidos (granulares o similares) serán también tratadas como cargas muertas. \\
    
Se consideran como cargas muertas los siguientes conjuntos:

\begin{enumerate}[label=\alph*) ]
    \item Peso de la estructura que considera los elementos estructurales pesados: vigas, techumbre, muros, tabiques, etc., y los no estructurales pesados: peso de las terminaciones (cielos falsos, estuco, terminaciones, etc.).
    
    \item Carga suspendida. Todas las cargas colgantes permanentes tales como puentes de cañerías, bandejas de soportes de cables eléctricos, luminarias, etc.

    % \item La acumulación de polvo en la estructura será considerada como carga permanente, con un peso específico de 1400 kg/m3 y un talud de 1:1,5 (V:H).
\end{enumerate}

\subsection{Cargas de uso}

Como carga de uso se considera lo establecido en la NCh 1537. A continuación se listan las cargas aplicables al caso:

\begin{table}[H]
  \centering
  \caption{Cargas de uso aplicables a la estructura}
        \begin{tabular}{|l|l|c|}
        \hline
        \multicolumn{1}{|c|}{\textbf{Tipo de edificio}} & \multicolumn{1}{c|}{\textbf{Descripción de uso}} & \multicolumn{1}{p{5.355em}|}{\textbf{Carga de uso [kPa]}} \bigstrut\\
        \hline
        \multirow{4}[2]{*}{Viviendas} & Áreas de uso general & 2 \bigstrut[t]\\
              & Dormitorios y buhardillas habitables & 2 \\
              & Entretecho con almacenaje & 1,5 \bigstrut[b]\\
        \hline
        \end{tabular}%
  \label{carga-uso}%
\end{table}%

\subsection{Carga de nieve}

No se considera carga de nieve, debido a la ubicación geográfica de la vivienda.

\subsection{Carga de viento}

Para el cálculo de fuerzas debidas a la acción del viento se aplicará lo especificado en la norma NCh 432. Para ello se suponen los casos W+ y W-, con viento en sentido positivo y negativo de la dirección X, respectivamente. 

\insertimage[\label{viento}]{img/geometriaviento}{width=11cm}{Coeficiente de forma a considerar para cargas de viento.}