\newpage
\section{Conclusiones}

Con respecto al diseño al corte de los muros, se puede apreciar que dada la baja demanda de estos la armadura mínima horizontal resiste todas las solicitaciones, como también lo hace el muro de albañilería por si solo. Teniendo en cuenta las consideraciones de la norma y sus requerimientos mínimos se concluye que el espaciamiento entre escalerillas es de 3 hiladas para todos los muros, teniendo algunas variaciones en los bordes de ventanas y puertas.\\

Con respecto a las solicitaciones que recibe la cadena, se concluye que la alternativa de una cercha de madera es bastante conveniente, esto pues, su bajo peso y por consecuencia sus solicitaciones sobre la cadena son bajas. Esto lo hace una solución preferible por sobre una cercha elaborada con perfiles de acero galvanizado, dado que se tendrían mayores esfuerzos sobre la cadena aumentando por lo tanto las solicitaciones en los muros. \\

Con respecto al análisis a flexo-compresión se puede observar en los diagramas resultantes que todos los muros respondieron de buena manera ante las demandas. Para el caso del análisis en el plano por lo general se tiene una excesiva resistencia en comparación a las solicitaciones, sin embargo en el caso del análisis fuera del plano algunos muros tuvieron una resistencia levemente superior a las solicitaciones, como el caso de las geometrías 1, 6 y 9 (Figuras \ref{geom1} y \ref{geom9} respectivamente). Con respecto a la consideraciones tomadas para el diseño de los muros es importante destacar que no se consideró la armadura en compresión, asumiendo que la mayoría de este esfuerzo es tomado por la albañilería en sí. Se redujeron las solicitaciones sísmicas en un 50 según lo estipulado en el apartado 5.3.2 de la norma NCh1928. Por otro lado es importante destacar que el momento fuera del plano fue tributado utilizando un espaciamiento máximo de 84cm, 6 veces el espesor del muro. \\

En cuanto a las deformaciones, tal como se mencionó en el capítulo anterior, en todos los muros se obtuvieron desplazamientos bajo el límite de la norma. De los resultados se desprende que el mayor desplazamiento en el eje y corresponde al peso propio y la combinación C1 (PP) y el mayor desplazamiento en el eje x corresponde al sismo en y. El mismo comportamiento se puede apreciar en el gráfico de drift de nodos (Figura \ref{drift}). Cabe destacar que estos desplazamientos y drifts fueron obtenidos utilizando un espectro elástico, sin reducir, tal como se estipula en el apartado 5.3.4 de la norma.