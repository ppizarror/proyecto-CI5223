\newpage
\section{Estructuración}

\subsection{Esquema de la estructura}

La estructura a diseñar consiste en una casa de una planta, con un total de 85.7 $m^2$ de superficie habitable. La Figura \ref{esqplanos} ilustra las dimensiones de la estructura idealizada.

\begin{images}[\label{esqplanos}]{Plano tipo de la casa.}
    \addimage{esqplanta}{width=7cm}{Esquema en planta.}
    \addimage{esqelev}{width=7cm}{Esquemas en elevación.}
\end{images}

Dado que la distancia de los ladrillos son discretos, se modificó el largo de los muros con tal de poder teselar \footnote{El teselado hace la referencia a una regularidad o patrón de figuras que recubren una superficie plana que cumplen los siguientes requisitos: no queden espacios entre las figuras y no existan superposiciones.} el muro sin tener que cortar los elementos. En las Tablas \ref{tab:tablalargos} y \ref{tab:tablaaltomuro} se detalla el largo modificado de cada muro, junto con el número de ladrillos a ubicar en el correspondiente largo y alto. Se consideró un total de $1\ cm$ de recubrimiento de mortero, tanto en el sentido horizontal como vertical.

\begin{table}[H]
  \centering
  \caption{Distancias finales de cada muro.}
  \itemresize{1.0}{
  
    \begin{tabular}{ccccc}
    \hline
    \textbf{Dimensión planta} & \textbf{Largo [cm]} & \textbf{Teselado} & \textbf{Ladrillos finales} & \textbf{Largo modificado (cm)} \bigstrut\\
    \hline
    L1    & 820   & 27.3  & 28    & 840 \bigstrut[t]\\
    L2    & 315   & 10.5  & 11    & 330 \\
    L3    & 505   & 16.8  & 17    & 510 \\
    L4    & 1000  & 33.3  & 34    & 1020 \\
    L5    & 390   & 13.0  & 13    & 390 \\
    L6    & 220   & 7.3   & 8     & 240 \\
    L7    & 225   & 7.5   & 8     & 240 \\
    L8    & 285   & 9.5   & 10    & 300 \\
    L9    & 105   & 3.5   & 4     & 120 \\
    L10   & 120   & 4.0   & 4     & 120 \\
    L11   & 90    & 3.0   & 3     & 90 \\
    L12   & 154   & 5.1   & 5     & 150 \bigstrut[b]\\
    \hline
  \end{tabular}
    }
  \label{tab:tablalargos}
\end{table}

\begin{table}[H]
  \centering
  \caption{Distancias alturas en cada muro.}
   \itemresize{1.0}{
  \begin{tabular}{ccccc}
    \hline
    \textbf{Dimensión alturas} & \textbf{Largo [cm]} & \textbf{Teselado} & \textbf{Ladrillos finales} & \textbf{Altura final (cm)} \bigstrut\\
    \hline
    Muros & 260   & 25    & 25    & 260 \bigstrut\\
    \hline
  \end{tabular}
  }
  \label{tab:tablaaltomuro}
\end{table}

Con las dimensiones planteadas y considerando un espesor de muros igual al ancho de las unidades de albañilería (14 centímetros) con factor de corrección del material $FC=1$ (cerámico hecho a mano) se obtuvo la densidad de muros de la Tabla \ref{tabdensidad}. Este valor permite obtener un daño leve frente a demandas sísmicas \footnote{La experiencia Chilena establece que para una densidad de muros superior al $1\%$ se obtiene un nivel de daño bajo frente a sismos. Este porcentaje se ha obtenido de manera empírica tras el análisis del comportamiento de viviendas Chilenas.}.

\begin{table}[H]
  \centering
  \caption{Densidad de muros.}
  \begin{tabular}{cc}
    \hline
    \textbf{Eje de análisis} & \boldmath{}\textbf{$d_1$}\unboldmath{} \bigstrut\\
    \hline
    Eje x & 4.0\% \bigstrut[t]\\
    Eje y & 4.6\% \bigstrut[b]\\
    \hline
  \end{tabular}
  \label{tabdensidad}
\end{table}

\subsection{Tipo de estructuración}

El tipo de estructuración es del tipo muro de albañilería armada. En este sentido el muro tiene la función de transmitir cargas tipo gravitacionales (compresión) a las fundaciones y resistir cargas cortantes, tracciones y compresiones por flexión en en caso de un sismo. \\

De acuerdo a lo estipulado en la NCh433 Tabla 5.1 y Tabla 6.1 se tienen los siguientes parámetros de modificación de acuerdo al tipo de estructuración, en donde $R$ y $R_o$ corresponden a factores de reducción e $I$ es el nivel de importancia de la estructura.

\begin{table}[H]
  \centering
    \begin{tabular}{|c|c|}
    \hline
    \textbf{Parámetro} & \textbf{Valor} \bigstrut\\
    \hline
    R     & 4 \bigstrut\\
    \hline
    $R_o$ & 4 \bigstrut\\
    \hline
    I     & 1 \bigstrut\\
    \hline
    \end{tabular}%

\end{table}%

\subsection{Prediseño}

Para el prediseño se consideraron todos los muros interiores y exteriores de la estructura. Sin embargo el tímpano no se considerará dentro del modelo resistente estructural, esto debido a las recomendaciones vistas en el diseño estructural en albañilería. \\

En la Figura \ref{estructuracion} se ilustra el prediseño realizado en ETABS, donde es posible apreciar los elementos considerados, el meshado de muros y la viga cadena.

\insertimage[\label{estructuracion}]{estructuracion}{width=12cm}{Modelo estructural de la casa en ETABS.}

\newpage
El modelo a trabajar se asimila, en general, a la estructura de la Figura \ref{casitanachoql}.

\insertimage[\label{casitanachoql}]{renelagosengineers}{width=8cm}{Vivienda modelada con estructuración similar, hecha en base a muro de albañilería, cadena de hormigón y cercha de techo, con envigado en madera.}