\newpage
\section{Materiales}
A continuación, se entrega un detalle de los materiales a utilizar junto con su normativa asociada.
    \subsection{Albañilería Armada}
    La albañilería armada se compondrá de los materiales expuestos a continuación.
        \subsubsection{Ladrillo Cerámico}
        Se utilizará un ladrillo cerámico \quotes{Santiago 9E}.según clasificación de la NCh169.Of2001, el material de este ladrillo no se degrada con el tiempo y es de alta resistencia al fuego.\\
        \begin{images}[\label{imagenmultiple}]{Ladrillo cerámico Santiago 9E.}
            \addimage{img/ladrillo9e}{width=6.5cm}{Ladrillo convencional.}
            \addimage{img/ladrillomitad}{width=6.5cm}{Ladrillo mitad 9E.}
        \end{images}
        
        Dentro de sus principales características se encuentran:
        \begin{itemize}
            \item Resistencia a la compresión: $\geq 150 \quad kg/cm^2$
            \item tramitación térmica: $9-19 \quad W/m^{2}K$
            \item Adherencia al mortero: $\geq4,0 \quad kg/cm^2$
            \item Resistencia al Fuego: $F > 120$
            \item Aislación Acústica: $\geq 45\quad dB$ %8=========D%
            \item Absorción Humedad: $ < 14\%$
        \end{itemize}
        
        Con respecto a sus dimensiones se tiene que:
        \begin{itemize}
            \item Dimensiones en cm: 29x14x9,4
            \item Peso: $>3.35\quad Kg$
            \item Unidades/m2 (cantería 1.5 cm): $29.8$
            \item Hiladas en 1 m de altura, cantería 1.5 cm: $9 unidades$ 
        \end{itemize}
        
        \subsubsection{Mortero de Junta}
        De acuerdo a las Especificaciones para morteros de junta - ASTM C270, se hará uso de un mortero de cemento de albañilería entre juntas Tipo N, con las siguientes características:
        \begin{itemize}
            \item Resistencia mediana ($\geq 5.3 \quad MPa$).
            \item Para uso general.
            \item Muros resistentes interiores y exteriores.
            \item Enchapados de albañilería.
            \item Tiene más capacidad de flexión que morteros de alta resistencia.
            \item Exposición a clima severo.
        \end{itemize}
        
        \subsubsection{Barras de Refuerzo}
        Para la conformación de los muros de albañilería se necesitarán dos tipos de barras de refuerzo, Horizontales y Verticales.
            \begin{itemize}
                \item Barras Verticales\\
                Las Barras Verticales o Barras con resalte son idénticas a las utilizadas en elementos de hormigón armado, de acuerdo a lo estipulado en la norma NCh204.
                \item Barras Horizontales
                \begin{itemize}
                    \item Se tienen Escalerillas que van dispuestas en las juntas horizontales del muro, también Armaduras Electro-soldadas para la cadena, según lo establecido en las normas NCh1173, NCh1174, NCh218 y NCh219.
                    \item Para las escalerillas, su diámetro máximo se encuentra limitado por el espesor de la junta.
                    \item Por último, su uso está limitado en función de la demanda sobre el elemento, de acuerdo a lo establecido en la NCh2123.
                \end{itemize}
                
            \end{itemize}
        \subsubsection{Hormigón de Relleno (Grout)}
        El Hormigón de relleno se utiliza para el llenado de huecos en que se ubican las barras de refuerzo verticales.
        Dentro de sus propiedades requeridas de acuerdo a la norma NCh1928 se encuentran:
        \begin{itemize}
            \item Alto descenso de cono: $\geq 18 \quad cm$
            \item Resistencia a la compresión:$\geq 17.5 \quad MPa$
        \end{itemize}
        
        Las alturas máximas de llenado deben cumplir:
        \begin{table}[H]
          \centering
          \caption{Requerimientos para las altura de llenado}
          \begin{tabular}{|p{10.28em}|c|}
            \hline
            \textbf{Menor dimensión del hueco de las unidades en cm} &
              \multicolumn{1}{p{9.11em}|}{\textbf{Alura máxima del muro a llenar en cm}}
              \bigstrut\\
            \hline
            \multicolumn{1}{|c|}{5} &
              30
              \bigstrut[t]\\
            \multicolumn{1}{|c|}{6} &
              120
              \\
            Mayor o igual que 12 &
              240
              \bigstrut[b]\\
            \hline
          \end{tabular}
          \label{tab:dasd}
        \end{table}
            
    \subsection{Otros elementos}
    Otros elementos a considerar son:
    \begin{itemize}
        \item Hormigón Armado: para la conformación de la cadena de la vivienda.
        \item Tabiquería: la cual se compone de montantes, soleras, volcanita,y conexiones eléctricas.
        \item Madera: utilizada para elaboración de la cercha de techo, la cual descansará sobre la cadena de la estructura.
    \end{itemize}
    