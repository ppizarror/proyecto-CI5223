\newpage
\section{Combinaciones}
Los elementos estructurales serán diseñados para aquella combinación de cargas que genere la condición más desfavorable, en cuanto a su resistencia límite requerida. \\

Los estados de carga serán modificados por factores de mayoración o minoración, si corresponde, de acuerdo a la combinación general de cargas indicada en la norma NCh3171. Se tendrán dos casos:

\begin{itemize}
    \item Para el hormigón armado se considerará lo indicado en el apartado 9.1.1. de combinaciones básicas, según el método de los estados límite (LRFD), en que las combinaciones a utilizar serán:
        
        \newcommand{\caquita}{0.8cm}    
        \hspace{\caquita} C1          $=1,4PP$
        
        \hspace{\caquita} C2          $=1,2PP+1,6SC$
        
        \hspace{\caquita} C3.1        $=1,2PP+SC\pm1,4E_x$
        
        \hspace{\caquita} C3.2        $=1,2PP+SC\pm1,4E_y$
        
        \hspace{\caquita} C4.1        $=0,9PP\pm1,4E_x$
        
        \hspace{\caquita} C4.2        $=0,9PP\pm1,4E_y$

    \item Para la albañilería se considerará lo indicado en el apartado 9.2.1, de combinaciones de cargas nominales que se usan en el método de tensiones admisibles (ASD), en que las combinaciones a utilizar serán:
    
        \hspace{\caquita} C1          $=PP$
        
        \hspace{\caquita} C2          $=PP+SC$
        
        \hspace{\caquita} C3.1        $=PP+SC\pm E_x$
        
        \hspace{\caquita} C3.2        $=PP+SC\pm E_y$
        
        \hspace{\caquita} C4.1        $=PP\pm E_x$
        
        \hspace{\caquita} C4.2        $=PP\pm E_y$\\
\end{itemize}